\documentclass[%
	11pt,
	a4paper,
	utf8,
	%twocolumn
		]{article}	

\usepackage{style_packages/podvoyskiy_article_extended}


\begin{document}
\title{Практика использования и \\наиболее полезные конструкции языка Rust}

\author{}

\date{}
\maketitle

\thispagestyle{fancy}

\tableofcontents

\section{Ресуры по языку Rust}

\url{https://www.rust-lang.org/tools}

\url{https://doc.rust-lang.org/book/}

\url{https://doc.rust-lang.org/stable/rust-by-example/}

\section{Установка Rust}

Установить Rust проще всего с помощью утилиты \texttt{rustup} -- это установщик языка и менеджер версий. Для операционной системы Windows можно скачать \texttt{rustup-init.exe} со страницы проекта \url{https://www.rust-lang.org/learn/get-started}

Установить Rust на Linux можно так
\begin{lstlisting}[
style = bash,
numbers = none
]
$ curl https://sh.rustup.rs -sSf | bash
...
Current installation options:


default host triple: x86_64-unknown-linux-gnu
default toolchain: stable (default)
profile: default
modify PATH variable: yes

1) Proceed with installation (default)
2) Customize installation
3) Cancel installation
>1

info: profile set to 'default'
info: default host triple is x86_64-unknown-linux-gnu
info: syncing channel updates for 'stable-x86_64-unknown-linux-gnu'
...
\end{lstlisting}

Rust часто обновляется и чтобы получить последнюю версию, можно воспользоваться командной \texttt{rustup update}.

Собрать проект и обновить его зависимости можно с помощью утилиты \texttt{cargo}
\begin{lstlisting}[
style = bash,
numbers = none
]
cargo build  # build your project
cargo run  # cargo run
cargo test  # test project 
cargo doc  # build documentation for your project
cargo publish  # publish a libarary to crates.io
\end{lstlisting}

То есть \texttt{cargo} знает, как превратить Rust-код в исполняемый бинарный файл, а также может управлять процессом загрузки и компиляции проектных зависимостей.

\section{Вводные замечания}

\emph{Система владения} устанавливает \emph{время жизни} каждого значения, что делает ненужным сборку мусора в ядре языка и обеспечивает надежные, но вместе с тем гибкие интерфейсы для управления такими ресурсами, как сокеты и описатели файлов. Передеча (move) позволяет передавать значение от одного владельца другому, а заимствование (borrowing) -- использовать значение временно, не изменяя владельца.  

Rust -- типобезопасный язык. Но что понимается под типобезопасностью? Ниже приведено определение <<неопределенного поведения>> из стандарта языка С 1999 года, известного под названием <<C99>>: \emph{неопределнное поведение -- это поведение, являющееся следствием использования непереносимой или некорректной программной конструкции либо некорректных данных, для которого в настоящем Международном стандарте нет никаких требований}.

Рассмотрим следующую программу на С
\begin{lstlisting}[
style = c_cpp,
numbers = none
]
int main(int argc, char **argv) {
    // объявление одноэлементного массива беззнаковых длинных целых чисел
    unsigned long a[1];
    // обращение к 4-ому элементу массива; индекс, нарушает границу диапазона
    a[3] = 0x7ffff7b36cebUL; 
    return 0;
}
\end{lstlisting}

Эта программа обращается к элементу за концом массива \texttt{a}, поэтому согласно С99 ее \emph{поведение не определно}, т.е. она может делать все что угодно. {\color{red}<<Неопределенная>> операция не просто возвращает неопределнный результат, она дает программе карт-бланш на \emph{произвольное выполнение}(!).}

С99 предоставляет компилятору такое право, чтобы он мог генерировать более быстрый код. Чем возлагать на компилятор ответственность за обнаружение и обработку странного поведения вроде выхода за конец массива, стандарт предполагает, что программист должен позаботиться о том, чтобы такие ситуации никогда не возникали.

Если программа написана так, что ни на каком пути выполнения \emph{неопределенное выполнение невозможно}, то будем говорить, что программа \emph{корректна} (well defined).

Если встроенные в язык проверки \emph{гарантируют корректность программы}, то будем называть язык \emph{типобизопасным} (type safe).

Тщательно напсанная программа на C или C++ может оказаться типобезопасной, {\color{red}но ни C, ни C++ не является типобезопасным языком}: в приведенном выше примере нет ошибок типизации, и тем не менее она демонстрирует неопределенное поведение. С другой стороны, {\color{blue}Python -- \emph{типобезопасный} язык}, его интерпретатор тратит время на обнаружение выхода за границы массива и обрабатывает его лучше, чем компилятор С.

\section{Начало работы}

Создать проект на Rust можно командной \texttt{cargo new <project\_name>}
\begin{lstlisting}[
style = bash,
numbers = none	
]
$ cargo new hello  # создать проект hello
$ tree 
.
hello/
  Cargo.toml
  src/
    main.rs
$ cd hello
$ cargo run  # запустить проект
   Compiling hello v0.1.0 (/home/kosyachenko/Projects/GARBAGE/rust_projects/hello)
Finished dev [unoptimized + debuginfo] target(s) in 0.42s
Running `target/debug/hello`
Hello, world!
# Дерево проекта изменилось
$ tree
.
Cargo.lock  # артефакт
Cargo.toml
src/
  main.rs
target/  # артефакт
  CACHEDIR.TAG
  debug/
    build
    deps/
      hello-27...
      hello-27...d
    examples/
    hello
    hello.d
    incremental/
      hello-imy.../
        s-ghim...
\end{lstlisting}

В основном каталоге имеется файл \texttt{Cargo.toml}, содержащий описание метаданных проекта, таких как имя проекта, его версия и его зависимости. Исходный код попадает в директорию \texttt{src}.

Выполнение команды \texttt{cargo run} привело также к добавлению к проекту новых файлов. Теперь у нас в основном каталоге проекта есть файл \texttt{Cargo.lock} и каталог \texttt{target}. В \texttt{Cargo.lock} указываются конкретные номера версий всех зависимостей, чтобы будущие сборки составлялись точно также, как и эта, пока содержимое \texttt{Cargo.toml} не изменится.

\subsection{Первая программа на Rust}

Нужно как обычно с помощью \texttt{cargo new hello} создать новый проект. Перейти в созданную директорию проекта и в файле \texttt{main.rs} директории \texttt{src} написать следущее
\begin{lstlisting}[
title = {\sffamily ./src/main.rs},
language = C++,
numbers = none
]
fn greet_world() {
    println!("Hello, world!");
    let southern_germany = "Germany";
    let japan = "Japan";
    let regions = [southern_germany, japan];
    
    for region in regions.iter() {
        println!("{}", &region);
    }
}

fn main() {
    greet_world();
}
\end{lstlisting}

Восклицательный знак свидетельствует об использовании \emph{макроса}. Для операции присваивания в Rust, которую правильнее было бы называть \emph{привязкой переменной}, используется ключевое слово \texttt{let}. Поддержка Unicode предоставляется самим языком. 

Для \emph{литералов массива} используются \emph{квадратные скобки}. Для возврата итератора метод \texttt{iter()} может присутствовать во многих типах. Амперсанд <<заимствует>> \texttt{region} так, чтобы доступ предоставлялся \emph{только для чтения}.

Строки ганатировано получают кодировку UTF-8.








% Источники в "Газовой промышленности" нумеруются по мере упоминания 
\begin{thebibliography}{99}\addcontentsline{toc}{section}{Список литературы}
	\bibitem{koltzov-c-lang:2019}{ \emph{Кольцов Д.М.} Си на примерах. Практика, практика и только практика. -- СПб.: Наука и Техника, 2019. -- 288 с.}
\end{thebibliography}

%\listoffigures\addcontentsline{toc}{section}{Список иллюстраций}

\lstlistoflistings\addcontentsline{toc}{section}{Список листингов}

\end{document}
